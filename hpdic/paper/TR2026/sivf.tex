\documentclass[sigconf,nonacm]{acmart}

\usepackage{amsmath}
\usepackage{mathtools}
% \usepackage{amsthm}
\usepackage{algorithmic}
\usepackage{algorithm}

\usepackage{graphicx}      % 用于插入图片
\usepackage{subcaption}    % 关键!用于 \begin{subfigure}
\usepackage{booktabs}      % 关键!用于表格里的 \toprule, \midrule
\usepackage{amsmath}       % 用于数学公式
\usepackage{xcolor}        % 颜色支持

% if you use cleveref..
\usepackage[capitalize,noabbrev]{cleveref}

\newcommand{\don}[1]{\textcolor{blue}{[Dongfang: #1]}}

%%%%%%%%%%%%%%%%%%%%%%%%%%%%%%%%
% THEOREMS
%%%%%%%%%%%%%%%%%%%%%%%%%%%%%%%%
\theoremstyle{plain}
\newtheorem{theorem}{Theorem}[section]
\newtheorem{proposition}[theorem]{Proposition}
\newtheorem{lemma}[theorem]{Lemma}
\newtheorem{corollary}[theorem]{Corollary}
\theoremstyle{definition}
\newtheorem{definition}[theorem]{Definition}
\newtheorem{assumption}[theorem]{Assumption}
\theoremstyle{remark}
\newtheorem{remark}[theorem]{Remark}

\usepackage[textsize=tiny]{todonotes}

\usepackage{xspace}
\newcommand{\ours}{SIVF\xspace}

\AtBeginDocument{%
  \providecommand\BibTeX{{%
    Bib\TeX}}}

\setcopyright{none}
\settopmatter{printacmref=false}
\renewcommand\footnotetextcopyrightpermission[1]{}
\pagestyle{plain} % 使用简单页面样式
\fancypagestyle{firstpagestyle}{
  \fancyhf{}
  \renewcommand{\headrulewidth}{0pt}
  \renewcommand{\footrulewidth}{0pt}
}

\begin{document}

\title{SIVF: Streaming Inverted File Indexing on GPUs}

\author{Dongfang Zhao}
\email{dzhao@uw.edu}
\affiliation{%
  \institution{HPDIC Lab}
  \city{University of Washington}
%   \state{WA}
  \country{USA}
}

\begin{abstract}

\end{abstract}%%
\maketitle

\section{Introduction}

\subsection{Motivation}
\label{sec:motivation}


Real-world vector search applications, such as real-time recommendation and fraud detection, increasingly operate on streaming data where timeliness is critical. These systems require a sliding window model where expired vectors must be evicted as new vectors arrive to maintain bounded memory usage.
However, existing GPU-accelerated approximate nearest neighbors (ANN) indices are predominantly optimized for write-once-read-many workloads. 

To quantify the gap between insertion and eviction performance, we conducted a benchmark using Faiss~\cite{johnson2019billion} on an NVIDIA RTX 6000 GPU with the SIFT1M dataset~\cite{jegou2011product} on the Chameleon testbed~\cite{keahey2020lessons}.
As illustrated in Figure~\ref{fig:motivation}, we observe a severe performance asymmetry. While inserting a batch of 10,000 vectors takes only 28.2 ms due to efficient GPU parallelism, evicting the same number of vectors incurs a latency of 212.7 ms. This constitutes a 7.6 times slowdown.
\don{TODO: Convert the benchmark into CPP calling local compilation of Faiss}

\begin{figure}[h!]
    \centering
    \includegraphics[width=0.9\linewidth]{figures/motivation_bar_chart.pdf}
    \caption{The High Cost of Deletion. While GPU parallelism accelerates vector insertion (28.2 ms), the lack of in-place eviction support causes deletion latency to increase to 212.7 ms. This represents a 7.6 times slowdown. This asymmetry makes data eviction the primary bottleneck in streaming scenarios.}
    \label{fig:motivation}
\end{figure}

This bottleneck stems from the architectural mismatch in current library designs. Since standard IVF (Inverted File) indices lack support for efficient in-place deletion on the GPU, evicting data necessitates a costly CPU-GPU roundtrip. The entire index structure must be copied back to the host, compacted on the CPU, and re-uploaded to the device. This IO-bound operation prevents the system from fully utilizing the GPU compute throughput and limits the maximum sustainable ingestion rate of the system.

\section*{Acknowledgment}
Results presented in this paper were obtained using the Chameleon testbed supported by the National Science Foundation.

\bibliography{ref}
\bibliographystyle{acm}

% \newpage
\appendix

\end{document}
\endinput
%%
%% End of file `sample-sigconf.tex'.
